\chapter{Results of inspection}

This chapter contains the results of the code inspection that we did on the assigned class and methods. All the points of the checklist reported in the assignment were checked, and we also found other bad practices not listed in the checklist.

\section{Notation}
\begin{itemize}
	\item The items of the checklist reported in the assignment will be referred as \textbf{C1}, \textbf{C2}, \ldots, while the general errors will be indicated with \textbf{Gen1}, \textbf{Gen2}, \ldots
	\item A specific line of code will be referred as follows: L.1234.
	\item An interval of lines of code will be referred as follows: L.1234$\sim$1289.
\end{itemize}

\section{Report}

\subsection{Checklist}

\begin{enumerate}
	\item \textbf{C11} L.50 L.51 L.53 L.55 L.58 L.60 L.144 L.146 L.147 L.285 L.289 L.291 L.329 L.350 L.360 L.439 L.443 L.445 L.485 L.506 L.516 \\ These lines contain \textit{if} statements containing only one statement, not surrounded by curly braces.
	\item \textbf{C12} L.264 L.294 L.336 L.418 L.448 L.492 \\ There are not blank lines between the end of a method and the beginning of the documentation of the following one.
	\item \textbf{C13} \\ For at least half of the code the line length exceed 80 characters.
	\item \textbf{C14} \\ L.62 L.75 L.90 L.111 L.134 L.144 L.160 L.204 L.212 L.216 L.273 L.283 L.288 L.294 L.305 L.315 L.319 L.330 L.337 L.370 L.427 L.442 L.448 L.449 L.459 L.469 L.474 L.486 L.493 \\ These lines exceed the length of 120 characters. Some of them are comments or part of the javadoc.
	\item \textbf{C17} L.161 \\ The \textit{else if} statement is not aligned with the beginning of the expression at the same level as the previous line.
	\item \textbf{C17} L.132 L.88 L.179 L.186 L.196 L.205 L.276 L.308 L.430 L.462 \\ The \textit{catch} statement is not aligned at the same level. This could be avoided either during the code development by proper checking.
	\item \textbf{C17} L.189 L.350 L.360 L.506 L.516 \\ The \textit{else} statement is not aligned at the same level. This could be avoided either during the code development by proper checking.
	\item \textbf{C18} \\ While methods are well explained (``field knowledge'' is still needed to understand), the class entirely lacks the description.
	\item \textbf{C19} L.96 L.97 L.273$\sim$280 \\ Four TODO comments are present in the code. The former are poorly explained and lack the removal time. The latter are better explained but still no date.
	\item \textbf{C24} \\ The package and import statements were found in the correct order as per the code inspection document.
	\item \textbf{C25} \\ The class or interface declarations order is followed correctly.
	\item \textbf{C27} L.224 L.378 \\ Both the \textit{switch} blocks are a copy paste in our service which paves way for duplicates. But this is quite preferable here rather than using different piece of code for the same logical implementation (i.e. for a better understanding).
	\item \textbf{C27} L.218 L.271 L.301 L.344 L.372 L.425 L.455 L.500 \\ All the methods are long more than 10 lines and some methods are even more than 20 lines. This should be optimized by using fewer conditional checks covering all the requirements.
	\item \textbf{C26} \\ All the methods are grouped by functionality rather than by scope or accessibility.
	\item \textbf{C33} L.313$\sim$319 L.352 L.435$\sim$438 L.467$\sim$474 L.508 \\ These lines contain declarations within the blocks the belong to, instead of being at the beginning.
	\item \textbf{C40} L.92 L.124 L.137 L.144$\sim$147 L.182 L.183 L.192 L.201 L.223 L.337 \\ We can find the inappropriate use of '==' and '!=' for object comparison many times in the code. The list of the lines:
	\begin{itemize}
	\vspace*{-0.25cm}
		\item \emph{L.92}: LinkedList
		\item \emph{L.124}: String
		\item \emph{L.137}: List
		\item \emph{L.144$\sim$157}: Timestamp
		\item \emph{L.182}: GenericValue
		\item \emph{L.183}: String
		\item \emph{L.192}: List
		\item \emph{L.201}: TechDataCalendar
		\item \emph{L.223}: Double
		\item \emph{L.377}: Double
	\vspace*{-0.05cm}
	\end{itemize}
	\item \textbf{C42} \\ The error messages are comprehensive and are in-line as per the code inspection document.
	\item \textbf{C44} L.224 L.378 \\ Two switch loops that could be considered brutish programming. Anyway since the switch loop cycles constants, I think that adopting a String vector to replace it, would damage the readability.
	\item \textbf{C54} \\ All switch statements are addressed by a break.
	\item \textbf{C55} L.224 L.378 \\ There is no default branch for any of the switch statements. The solutions we suggest for this problem is to leave the code undisturbed, because all the days are present in the switch cases such that it executes any one of the switch case and so no default is needed. Without knowing the dayStart or dayEnd it would be wrong to define a default branch here.
	
\subsection{General}
Other inaccuracies found during the code inspection:

	\item \textbf{Gen1} L.261 L.284 L.317 L.328 L.415 L.438 L.471 L.483 \\ No import statement used for Integer. Import that has to be used here was: java.lang.Integer or import java.lang.* can be used.
	\item \textbf{Gen2} L.221 L.287 L.356 L.375 L.441 L.472 L.512 \\ No import for Double was done: java.lang.Double or import java.lang.* could have solved the issue.
	\item \textbf{Gen3} L.418 L.448 L.264 L.394 L.336 \\ TechDataCalendar was spelled wrong here within the comments. The final letter is missing.
\end{enumerate}