\chapter{Description of the code}

\section{Revision history}
\begin{table}[h]
	\centering
	\begin{tabular}{|c|c|c|c|}
		\hline
		\textbf{Version} & \textbf{Date} & \textbf{Authors} & \textbf{Summary} \\
		\hline
		1.0 & 05/02/2017 & Fabiani, Manivannan, Pozzolini & Initial release  \\
		\hline
	\end{tabular}

	\caption{Changelog of this document}
	\label{tab:revision_history}
\end{table}

\section{Assigned class: TechDataServices}
The TechDataServices.java is the class given for us for code inspection. Before executing our code inspection, a basic understanding of the code
was required. From our study of the code, the TechDataServices.java class performs the following operations in the below mentioned sequence:
\begin{itemize}
	\item Retrieves some RoutingTasks selected by Name or MachineGroup ordered by Name.
	\item Checks if there are no two routing tasks with the same SeqId are valid at the same period.
	\item Gets the techDataCalendar for a routingTask, if there is a routingTask associated with no MachineGroup the DEFAULT TechDataCalendar is returned.
	\item Finds the fisrt day in the TechDataCalendarWeek where capacity != 0, beginning at dayStart (dayStart included).
	\item Requests the remain capacity available for dateFrom in a TechDataCalendar.
If the dateFrom is not in an available TechDataCalendar period,it returns zero value.
	\item Moves a day in TechDataCalendar, by producing the Timestamp for the beginning of the next day available and its associated capacity.
If the dateFrom is not in an available TechDataCalendar period,it returns the next day available.
	\item Moves forward in a TechDataCalendar, starting from the dateFrom and moves forward only on available period. If the dateFrom is not an available TechDataCalendar period, the startDate is the begining of the next available day.
	\item Finds the last day in the TechDataCalendarWeek where capacity != 0, ending at dayEnd (dayEnd included).
	\item Requests the remain capacity available for dateFrom in a TechDataCalendar. If the dateFrom is not in an available TechDataCalendar period, it returns zero value.
	\item Moves a day in TechDataCalendar, by producing the Timestamp for the end of previous day available and its associated capacity.
If the dateFrom is not in an available TechDataCalendar period,it returns the previous day available.
	\item Moves backward in a TechDataCalendar, starting from the dateFrom and moves backward only on available period. If the dateFrom is not an available TechDataCalendar period, the startDate is the end of previous available day.
\end{itemize}

These are just some routing checks that are performed and the methods of this class is implicitly used or called by other classes (from other packages).