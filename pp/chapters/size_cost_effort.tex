\chapter{Project size, cost and effort estimation}

\section{Size estimation}

\subsection{Internal logic files (ILFs)}
The application includes a number of ILFs that will be used to store the information about users, payments, reservations, rides, cars, parking slots and discounts. Their weights will be calculated according to their structure's complexity. Most of them have a simple structure consisting in two or three fields, so we will assign simple weights to these entities.

Reservation has a more complex structure: it saves the user who creates the reservation, the car involved, the timestamp of the reservation and its status (RESERVED, RIDING, EXPIRED, COMPLETED, where COMPLETED means that the driving process has been successfully concluded), so we will assign a medium weight to this entity. The same weight will be applied to RIDE, that saves the references to the reservation and the user, the timpestamp of the ride's beginning, and its status (RIDING, COMPLETED).

The most complex entity we have is USER, that saves all the personal infos and contacts (used by the notification component), its balance, and the possibly empty references to the ongoing ride and relative reservation. We will then assign a complex weight to it.

\begin{table}[H]
	\centering
	\begin{tabular}{|l|l|l|}
		\hline
		ILF & Complexity & FPs \\
		\hline
		RESERVATION(user, car, time, status) & Medium & 10 \\
		RIDE(user, resv, starttime, status) & Medium & 10 \\
		USER(data, balance, resv, ride) & Complex & 15 \\
		PAYMENT(user, amount) & Simple & 7 \\
		CAR(id, loc, status) & Simple & 7 \\
		PARKING(id, car) & Simple & 7 \\
		DISCOUNT(id, spec) & Simple & 7 \\
		\hline
		\multicolumn{2}{|l|}{Total} & 63 \\
		\hline	
	\end{tabular}
\end{table}

\subsection{External Logic Files (ELFs)}
The system interacts with the licence office system, but as explained in the RASD we designed it as an external inquiry: we upload the licence provided by the user retrieving a STRING response. We can then skip the licence office in this section.
The system also interacts with the payment service, but this interaction works like an external inquiry, so it won't be calculated in this section.

\subsection{External Inputs (EIs)}
PowerEnJoy supports many kind of interactions, involving different entities, which are described below.
\begin{itemize}
	\item \emph{Register}: the registration involves just the user entity, but since it implies to interact with the licence office system, we will assign a medium weight to this operation.
	\item \emph{Login, logout, update licence}: these operations just involve the user entity, so we will assign them simple weights.
	\item \emph{Pay}: payment implies the interaction with the payment system through the payment gateway, plus updating the payment and user entities. We will then assign a medium weight to this operation.
	\item \emph{Reserve car, abort reservation}: these operations involve three entities (reservation, car, user) and are two of the most dangerous operations concerning the database integrity. Complex weights will then be assigned.
	\item \emph{Open car, end ride}: these operations involve four entities (reservation, car, user, ride) and as the previous two, these dangerous operations concerning the database integrity. Complex weights will then be assigned.
	\item \emph{Plug}: this operation is not trivial even if just two entities are involved, mid weight will be assigned.
	\item \emph{Ban user, add/remove car}: the operations of adding/removing a car or banning a user (possible only for admin users) will imply other operations in the real world, but according to the system its still a basic operation involving just one entity. Simple weight will be assigned.
\end{itemize}

\begin{table}[H]
	\centering
	\begin{tabular}{|l|l|l|l|}
		\hline
		EI & Entities involved & Complexity & FPs \\
		\hline
		Register & User & Middle & 4 \\
		Login/Logout & User & Simple & 3 \\
		Update licence & User & Simple & 3 \\
		Pay & Payment, User & Middle & 4 \\
		Reserve car & Reservation, Car, User & Complex & 6 \\
		Abort reservation & Reservation, Car, User & Complex & 6 \\
		Open car/End ride & Reservation, Car, User, Ride & Complex & 6 \\
		Plug & Car, Parking & Middle & 4 \\
		Ban user & User & Simple & 3 \\
		Add/remove car & Car & Simple & 3 \\
		\hline
		\multicolumn{3}{|l|}{Total} & 42 \\
		\hline	
	\end{tabular}
\end{table}

\subsection{External Inquiries (EQs)}
As specified by the function points guidelines, an inquiry is essentially a data retrieval
request performed by an user. PowerEnJoy supports a few interactions of this type that don’t require
complex computations:
\begin{itemize}
	\item \emph{Get available cars (zone)}: this operation retrieves the available cars given a zone (the visible area in the map). It just involves the car entity, simple weight will then be applied.
	\item \emph{Get payment history}: this operation retrieves the history of user's payments. It just involves just the payment entity, simple weight will then be applied.
	\item \emph{Get active reservation}: this operation retrieves, if there is one, the user's active reservation It involves reservation and user entities but its still a simple process, simple weight will then be applied.
	\item \emph{Get available discounts}: this operation retrieves the list of available discounts. It involves just the discount entity, simple weight will be applied.
\end{itemize}

\begin{table}[H]
	\centering
	\begin{tabular}{|l|l|l|l|}
		\hline
		EQ & Entities involved & Complexity & FPs \\
		\hline
		Get available cars (zone) & Car & Simple & 3 \\
		Get payment history & Payment & Simple & 3 \\
		Get active reservation & Reservation, User & Simple & 3 \\
		Get available discounts & Discount & Simple & 3 \\
		\hline
		\multicolumn{3}{|l|}{Total} & 12 \\
		\hline
	\end{tabular}
\end{table}

\subsection{External Outputs (EOs)}
As part of its normal behaviour, our system needs to communicate with the user outside the context of an inquiry in the following occasion:
\begin{itemize}
	\item \emph{Notify}: Despite how it may sound, notifying events is a quite complex operation in our system. There are different types of notification, different output flows (SMS, e-mail, push notification) and many events to be notified; a specific component (NotificationHelper) is designed to dispatch every notification according to its type and user settings.
\end{itemize}

\begin{table}[H]
	\centering
	\begin{tabular}{|l|l|l|l|}
		\hline
		EO & Entities involved & Complexity & FPs \\
		\hline
		Notify & User & Complex & 7 \\
		\hline
		\multicolumn{3}{|l|}{Total} & 7 \\
		\hline
	\end{tabular}
\end{table}

\subsection{Overall estimation}
The following table summarizes the results of our estimation activity:

\begin{table}[H]
	\centering
	\begin{tabular}{|l|l|}
		\hline
		Function type & Value \\
		\hline
		Internal Logic Files & 63 \\
		External Logic Files & 0 \\
		External Inputs & 42 \\
		External Inquiries & 12 \\
		External Outputs & 7 \\
		\hline
		Total & 124 \\
		\hline
	\end{tabular}
\end{table}
Considering Java Enterprise Edition as a development platform and disregarding the aspects concerning the implementation of the mobile applications (which can be thought as pure presentation with no business logic), we can estimate the total number of lines of code.

Depending on the conversion rate, we have a lower bound of \[KSLOC = 133 \times 51 = 6783\] and an upper bound of \[KSLOC = 133 \times 73 = 9709\]

\section{Cost and effort estimation}
In this section we are going to use the COCOMO II approach to estimate the
cost and effort needed to develop the PowerEnJoy application.

\subsection{Scale drivers}
In order to evaluate the values of the scale drivers, we refer to the following official COCOMO II table:

\begingroup
\captionsetup{skip=10pt}
\setlength{\LTleft}{-1.7cm plus -1fill}
\setlength{\LTright}{\LTleft}
\vspace*{0.4cm}
\begin{longtable}{|c|p{2cm}|p{2cm}|p{2cm}|p{2cm}|p{2cm}|p{2cm}|}
	\hline
	Scale factors & Very low & Low & Nominal & High & Very high & Extra high \\
	\hline
	PREC & thoroughly unprecedented & largely unprecedented & somewhat unprecedented & generally familiar & largely familiar & thoroughly familiar \\
	SF\(_j\) & 6.20 & 4.96 & 3.72 & 2.48 & 1.24 & 0.00 \\
	\hline
	FLEX & rigorous & occasional relaxation & some relaxation & general conformity & some conformity & general goals \\
	SF\(_j\) & 5.0 & 4.05 & 3.04 & 2.03 & 1.01 & 0.00 \\
	\hline
	RESL & little (20\%) & some (40\%) & often (60\%) & generally (75\%) & mostly (90\%) & full (100\%) \\
	SF\(_j\) & 7.07 & 5.65 & 4.24 & 2.83 & 1.41 & 0.00 \\
	\hline
	TEAM & very difficult interactions & some difficult interactions & basically cooperative interactions & largely cooperative & highly cooperative & seamless interactions \\
	SF\(_j\) & 5.48 & 4.38 & 3.29 & 2.19 & 1.10 & 0.00 \\
	\hline
	PMAT & Level 1 lower & Level  1 upper & Level 2 & Level 3 & Level 4  & Level 5 \\
	SF\(_j\) & 7.80 & 6.24 & 4.68 & 3.12 & 1.56 & 0.00 \\
	\hline
	
	\caption{Scale Factor values (SF\(_j\)) for COCOMO II Models}
\end{longtable}
\endgroup

A brief description for each scale driver:
\begin{itemize}
	\item \emph{Precedentedness}: this factor determines or reveals the level of exposure or experience in development of large scale projects or similar kind of projects that out team has done before. Since we have developed few projects like this, we can set this value to be Nominal.
	\item \emph{Development flexibility}: it determines the degree of flexibility in the development process with respect to the external specification and requirements. In our project, the functionalities and requirements are clear and well defined with no specific mention about the technology. Hence this value would be low.
	\item \emph{Architecture/Risk resolution}: it determines the level of awareness and reactivity with
respect to risks. Since we have an extremely good risk management plan, we consider this value to be very high.
	\item \emph{Team cohesion}: it determines if all the Stakeholders are able to work in a team and share same vision and commitment. Since our team is highly co-operative,
	the value is very high.
	\item \emph{Process maturity}: we have a done an extremely fair work to meet our goals successfully in this project. Since we had prior experience in successfully dealing these kind of projects, the value is set to Level 4.
\end{itemize}

The results of our evaluation is the following:

\begin{table}[H]
	\centering
	\begin{tabular}{|l|r|r|}
		\hline
		Scale Driver & Factor & Value \\
		\hline
		Precedentedness (PREC) & Nominal & 3.72 \\
		Development flexibility (FLEX) & Low & 4.05 \\
		Risk resolution (RESL) & Very high & 1.41 \\
		Team cohesion (TEAM) & Very high & 1.10 \\
		Process maturity (PMAT) & Level 4 & 1.56 \\
		\hline
		\multicolumn{2}{|l|}{Total} & 11.84 \\
		\hline	
	\end{tabular}
\end{table}

\subsection{Cost drivers}

\subsubsection{Product factors}

\begin{itemize}
	\item \emph{Required Software Reliability (RELY)}:
	\begin{itemize}
			\item[] The software application is developed in such a way that the main aim is to reserve and take a ride in the Cars in the city. Any malfunctioning could lead to important financial loss. Considering this, the RELY cost driver is set to high.
	\end{itemize}
\end{itemize}

\begin{table}[H]
	\hspace*{-1.7cm}
	\begin{tabular}{|p{2cm}|p{2cm}|p{2cm}|p{2cm}|p{2cm}|p{2cm}|p{2cm}|}
		\hline
		\multicolumn{7}{|c|}{RELY cost drivers} \\
		\hhline{|=======|}
		RELY descriptors & slightly inconvenience & easily recoverable losses & moderate recoverable losses & high financial loss & risk to human life & \\
		\hline
		Rating level & Very low & Low & Nominal & High & Very high & Extra high \\
		\hline
		Effort multipliers & 0.82 & 0.92 & 1.00 & 1.10 & 1.26 & n/a \\
		\hline
	\end{tabular}
\end{table}

\begin{itemize}
	\item \emph{Database size (DATA)}:
	\begin{itemize}
		 \item[] This factor considers the effective size of our database. We do'nt know this value exactly. But based on the lower and upper bound
values of the SLOC, which is 10.000-15.000 SLOC, we can estimate roughly that our system can reach a 3GB database size. Since it is
distributed over 10.000-15.000 SLOC, the ratio D/P (measured as testing DB bytes/program SLOC) is between 209 and 314,
resulting in the DATA cost driver being high.
	\end{itemize}
\end{itemize}

\begin{table}[H]
	\hspace*{-1.7cm}
	\begin{tabular}{|p{2cm}|p{2cm}|p{2cm}|p{2cm}|p{2cm}|p{2cm}|p{2cm}|}
		\hline
		\multicolumn{7}{|c|}{DATA cost drivers} \\
		\hhline{|=======|}
		DATA descriptors & & \(\frac{D}{P} < 10\) & \(10 \leq \frac{D}{P} < 100\) & \(100 \leq \frac{D}{P} < 1000\) & \(\frac{D}{P} \geq 1000\) & \\
		\hline
		Rating level & Very low & Low & Nominal & High & Very high & Extra high \\
		\hline
		Effort multipliers & n/a & 0.90 & 1.00 & 1.14 & 1.28 & n/a \\
		\hline
	\end{tabular}
\end{table}

\begin{itemize}
	\item \emph{Product complexity (CPLX)}:
	\begin{itemize}
		\item[] This factor is related to the complex logics involved in implementing the product as a whole.
Hence, we set it to very high according to the CPLX cost driver table.
	\end{itemize}
\end{itemize}

\begin{table}[H]
	\hspace*{-1.7cm}
	\begin{tabular}{|p{2cm}|p{2cm}|p{2cm}|p{2cm}|p{2cm}|p{2cm}|p{2cm}|}
		\hline
		\multicolumn{7}{|c|}{CPLX cost drivers} \\
		\hhline{|=======|}
		Rating level & Very low & Low & Nominal & High & Very high & Extra high \\
		\hline
		Effort multipliers & 0.73 & 0.87 & 1.00 & 1.17 & 1.34 & 1.74 \\
		\hline
	\end{tabular}
\end{table}

\begin{itemize}
	\item \emph{Developed for Reusability (RUSE)}:
	\begin{itemize}
		\item[] In our project, we use many individual piece of codes that can be made reusable for other services or functions.
Hence the RUSE cost driver is set to nominal.
	\end{itemize}
\end{itemize}

\begin{table}[H]
	\hspace*{-1.7cm}
	\begin{tabular}{|p{2cm}|p{2cm}|p{2cm}|p{2cm}|p{2cm}|p{2cm}|p{2cm}|}
		\hline
		\multicolumn{7}{|c|}{RUSE cost drivers} \\
		\hhline{|=======|}
		RUSE descriptors & & None & Across project & Across program & Across product line & Across multiple product \\
		\hline
		Rating level & Very low & Low & Nominal & High & Very high & Extra high \\
		\hline
		Effort multipliers & n/a & 0.95 & 1.00 & 1.07 & 1.15 & 1.24 \\
		\hline
	\end{tabular}
\end{table}

\begin{itemize}
	\item \emph{Documentation Match to Life-Cycle Needs (DOCU)}:
	\begin{itemize}
		\item[] This factor describes the relationship between the documentation and the application requirements. The product life-cycle needs are explicitly mentioned clearly in the documentation. Hence the DOCU cost driver is set to nominal.
	\end{itemize}
\end{itemize}

\begin{table}[H]
	\hspace*{-1.7cm}
	\begin{tabular}{|p{2cm}|p{2cm}|p{2cm}|p{2cm}|p{2cm}|p{2cm}|p{2cm}|}
		\hline
		\multicolumn{7}{|c|}{DOCU cost drivers} \\
		\hhline{|=======|}
		DOCU descriptors & Many life-cycle needs uncovered & Some life-cycle needs uncovered & Right sized to life-cycle needs & Excessive for life-cycle needs & Very excessive for life-cycle needs & \\
		\hline
		Rating level & Very low & Low & Nominal & High & Very high & Extra high \\
		\hline
		Effort multipliers & 0.81 & 0.91 & 1.00 & 1.11 & 1.23 & n/a \\
		\hline
	\end{tabular}
\end{table}

\subsubsection{Platform factors}

\begin{itemize}
	\item \emph{Execution Time Constraint (TIME)}:
	\begin{itemize}
		\item[] This factor describes the approximated value of CPU usage with respect to the hardware specifications. Our PowerEnJoy application has vast functionalities as a software and hence the TIME cost driver is set to be very high.
	\end{itemize}
\end{itemize}

\begin{table}[H]
	\hspace*{-1.7cm}
	\begin{tabular}{|p{2cm}|p{2cm}|p{2cm}|p{2cm}|p{2cm}|p{2cm}|p{2cm}|}
		\hline
		\multicolumn{7}{|c|}{TIME cost drivers} \\
		\hhline{|=======|}
		TIME descriptors & & & \(\leq 50\%\) use of available execution time & 70\% use of available execution time & 85\% use of available execution time & 90\% use of available execution time \\
		\hline
		Rating level & Very low & Low & Nominal & High & Very high & Extra high \\
		\hline
		Effort multipliers & n/a & n/a & 1.00 & 1.11 & 1.29 & 1.63 \\
		\hline
	\end{tabular}
\end{table}

\begin{itemize}
	\item \emph{Main Storage Constraint (STOR)}:
	\begin{itemize}
		\item[] This factor describes the approximated storage space with respect to the hardware specifications. Our PowerEnJoy application
has vast functionalities as a software. Keeping this in mind, the disk drives can store up to enough terabytes and hence the STOR cost driver is set to be high.
	\end{itemize}
\end{itemize}

\begin{table}[H]
	\hspace*{-1.7cm}
	\begin{tabular}{|p{2cm}|p{2cm}|p{2cm}|p{2cm}|p{2cm}|p{2cm}|p{2cm}|}
		\hline
		\multicolumn{7}{|c|}{STOR cost drivers} \\
		\hhline{|=======|}
		STOR descriptors & & & \(\leq 50\%\) use of available storage & 70\% use of available storage & 85\% use of available storage & 90\% use of available storage \\
		\hline
		Rating level & Very low & Low & Nominal & High & Very high & Extra high \\
		\hline
		Effort multipliers & n/a & n/a & 1.00 & 1.05 & 1.17 & 1.46 \\
		\hline
	\end{tabular}
\end{table}

\begin{itemize}
	\item \emph{Platform Volatility (PVOL)}:
	\begin{itemize}
		\item[] This factor describes the change in the basic or fundamental platform in which the system is designed. We do'nt change the platform often except for very few major releases or updated requested by the client. This will be done approximately for every 5months to be in sync with the latest evolving or trending technologies. Hence, the PVOL cost driver is set to nominal.
	\end{itemize}
\end{itemize}

\begin{table}[H]
	\hspace*{-1.7cm}
	\begin{tabular}{|p{2cm}|p{2cm}|p{2cm}|p{2cm}|p{2cm}|p{2cm}|p{2cm}|}
		\hline
		\multicolumn{7}{|c|}{PVOL cost drivers} \\
		\hhline{|=======|}
		PVOL descriptors & & Major change every 12~months; minor change every 1~month & Major change every 6~months;\newline minor change every 2~weeks & Major change every 2~months;\newline minor change every 1~week & Major change every 2~weeks;\newline minor change every 2~days & \\
		\hline
		Rating level & Very low & Low & Nominal & High & Very high & Extra high \\
		\hline
		Effort multipliers & n/a & 0.87 & 1.00 & 1.15 & 1.30 & n/a \\
		\hline
	\end{tabular}
\end{table}

\begin{itemize}
	\item \emph{Analyst Capability (ACAP)}:
	\begin{itemize}
		\item[] This factor describes the potential analysis that has been done with respect to the potential implementation in real world. Since we have done a regressive analysis, the ACAP cost driver is set to be high.
	\end{itemize}
\end{itemize}

\begin{table}[H]
	\hspace*{-1.7cm}
	\begin{tabular}{|p{2cm}|p{2cm}|p{2cm}|p{2cm}|p{2cm}|p{2cm}|p{2cm}|}
		\hline
		\multicolumn{7}{|c|}{ACAP cost drivers} \\
		\hhline{|=======|}
		ACAP descriptors & 15th percentile & 35th percentile & 55th percentile & 75th percentile & 90th percentile & \\
		\hline
		Rating level & Very low & Low & Nominal & High & Very high & Extra high \\
		\hline
		Effort multipliers & 1.42 & 1.19 & 1.00 & 0.85 & 0.71 & n/a \\
		\hline
	\end{tabular}
\end{table}

\begin{itemize}
	\item \emph{Programmer Capability (PCAP)}:
	\begin{itemize}
		\item[] This factor describes the ability of the programmer to do a work without much difficulty. Our project has not been implemented yet our programmers
have executed several projects like this successfully and hence the PCAP cost driver is set to be high.
	\end{itemize}
\end{itemize}

\begin{table}[H]
	\hspace*{-1.7cm}
	\begin{tabular}{|p{2cm}|p{2cm}|p{2cm}|p{2cm}|p{2cm}|p{2cm}|p{2cm}|}
		\hline
		\multicolumn{7}{|c|}{PCAP cost drivers} \\
		\hhline{|=======|}
		PCAP descriptors & 15th percentile & 35th percentile & 55th percentile & 75th percentile & 90th percentile & \\
		\hline
		Rating level & Very low & Low & Nominal & High & Very high & Extra high \\
		\hline
		Effort multipliers & 1.34 & 1.15 & 1.00 & 0.88 & 0.76 & n/a \\
		\hline
	\end{tabular}
\end{table}

\begin{itemize}
	\item \emph{Applications Experience (APEX)}:
	\begin{itemize}
		\item[] Our team members are quite experienced with this kind of project development and hence the  APEX cost driver is set to be high.
	\end{itemize}
\end{itemize}

\begin{table}[H]
	\hspace*{-1.7cm}
	\begin{tabular}{|p{2cm}|p{2.04cm}|p{1.96cm}|p{2cm}|p{2cm}|p{2cm}|p{2cm}|}
		\hline
		\multicolumn{7}{|c|}{APEX cost drivers} \\
		\hhline{|=======|}
		APEX descriptors & \(\leq\) 2 months & 6 months & 1 years & 3 years & 6 years & \\
		\hline
		Rating level & Very low & Low & Nominal & High & Very high & Extra high \\
		\hline
		Effort multipliers & 1.22 & 1.10 & 1.00 & 0.88 & 0.81 & n/a \\
		\hline
	\end{tabular}
\end{table}

\begin{itemize}
	\item \emph{Platform Experience (PLEX)}:
	\begin{itemize}
		\item[] Our team has a good and stable experience in Java EE platform and also a good knowledge about the integration with UI, Database and other tiers. Hence the PLEX cost driver is set to be high.
	\end{itemize}
\end{itemize}

\begin{table}[H]
	\hspace*{-1.7cm}
	\begin{tabular}{|p{2cm}|p{2.04cm}|p{1.96cm}|p{2cm}|p{2cm}|p{2cm}|p{2cm}|}
		\hline
		\multicolumn{7}{|c|}{PLEX cost drivers} \\
		\hhline{|=======|}
		PLEX descriptors & \(\leq\) 2 months & 6 months & 1 years & 3 years & 6 years & \\
		\hline
		Rating level & Very low & Low & Nominal & High & Very high & Extra high \\
		\hline
		Effort multipliers & 1.19 & 1.09 & 1.00 & 0.91 & 0.85 & n/a \\
		\hline
	\end{tabular}
\end{table}

\begin{itemize}
	\item \emph{Language and Tool Experience (LTEX)}:
	\begin{itemize}
		\item[] As we have mentioned before, since the knowledge of our programmers are good enough on this kind of project and Java EE platform, they possess a good standard of using tools in the development environment, server side and client side integration,etc. Hence the LTEX cost driver is set to be high.
	\end{itemize}
\end{itemize}

\begin{table}[H]
	\hspace*{-1.7cm}
	\begin{tabular}{|p{2cm}|p{2.04cm}|p{1.96cm}|p{2cm}|p{2cm}|p{2cm}|p{2cm}|}
		\hline
		\multicolumn{7}{|c|}{LTEX cost drivers} \\
		\hhline{|=======|}
		LTEX descriptors & \(\leq\) 2 months & 6 months & 1 years & 3 years & 6 years & \\
		\hline
		Rating level & Very low & Low & Nominal & High & Very high & Extra high \\
		\hline
		Effort multipliers & 1.20 & 1.09 & 1.00 & 0.91 & 0.84 & n/a \\
		\hline
	\end{tabular}
\end{table}

\begin{itemize}
	\item \emph{Personnel Continuity (PCON)}:
	\begin{itemize}
		\item[] This factor describes the personnel turnover annually. Since our project is a short term project, the PCON cost driver is set to be very low.
	\end{itemize}
\end{itemize}

\begin{table}[H]
	\hspace*{-1.7cm}
	\begin{tabular}{|p{2cm}|p{2cm}|p{2cm}|p{2cm}|p{2cm}|p{2cm}|p{2cm}|}
		\hline
		\multicolumn{7}{|c|}{PCON cost drivers} \\
		\hhline{|=======|}
		PCON descriptors & 48\%/year & 24\%/year & 12\%/year & 6\%/year & 3\%/year & \\
		\hline
		Rating level & Very low & Low & Nominal & High & Very high & Extra high \\
		\hline
		Effort multipliers & 1.29 & 1.12 & 1.00 & 0.90 & 0.81 & n/a \\
		\hline
	\end{tabular}
\end{table}

\subsubsection{Project factors}

\begin{itemize}
	\item \emph{Use of Software Tools (TOOL)}:
	\begin{itemize}
		\item[] Our application environment is complete and well integrated, so we will set this parameter as high.
	\end{itemize}
\end{itemize}

\begin{table}[H]
	\hspace*{-1.7cm}
	\begin{tabular}{|p{2cm}|p{2cm}|p{2cm}|p{2cm}|p{2cm}|p{2cm}|p{2cm}|}
		\hline
		\multicolumn{7}{|c|}{TOOL cost drivers} \\
		\hhline{|=======|}
		TOOL descriptors & edit, code, debug & simple, front-end, back-end CASE, little integration & basic life-cycle tools, moderately integrated & strong, mature life-cycle tools, moderately integrated & strong, mature, proactive life-cycle tools, well integrated with processes, methods, reuse & \\
		\hline
		Rating level & Very low & Low & Nominal & High & Very high & Extra high \\
		\hline
		Effort multipliers & 1.17 & 1.09 & 1.00 & 0.90 & 0.78 & n/a \\
		\hline
	\end{tabular}
\end{table}

\begin{itemize}
	\item \emph{Multisite Development (SITE)}:
	\begin{itemize}
		\item[] Our application is designed in such a way that it relies on wideband electronic communication at extremely good speeds (e.g. 3G, 4G) for connection. Hence the SITE cost driver is set to be very high.
	\end{itemize}
\end{itemize}

\begin{table}[H]
	\hspace*{-1.7cm}
	\begin{tabular}{|p{2cm}|p{2cm}|p{2cm}|p{2cm}|p{2cm}|p{2cm}|p{2cm}|}
		\hline
		\multicolumn{7}{|c|}{SITE cost drivers} \\
		\hhline{|=======|}
		SITE collocation descriptors & interna-tional & multi-city and multi-company & multi-city or multi-company & same city or metro area & same building or complex & fully collocated \\
		&&&&&& \\
		SITE communications descriptors & some phone, mail & individual phone, FAX & narrow band email & wideband electronic communication & wideband electronic communication, occasional video conference & interactive multimedia \\
		\hline
		Rating level & Very low & Low & Nominal & High & Very high & Extra high \\
		\hline
		Effort multipliers & 1.22 & 1.09 & 1.00 & 0.93 & 0.86 & 0.80 \\
		\hline
	\end{tabular}
\end{table}

\subsubsection{General factor}

\begin{itemize}
	\item \emph{Required Development Schedule (SCED)}:
	\begin{itemize}
		 \item[] The efforts was distributed or split equally in our project for all the documentation, yet there were certain time consuming process in analysing and development of the RASD and the DD documents for precision. Hence, the SCED cost driver is set to be high.
	\end{itemize}
\end{itemize}

\begin{table}[H]
	\hspace*{-1.7cm}
	\begin{tabular}{|p{2cm}|p{2cm}|p{2cm}|p{2cm}|p{2cm}|p{2cm}|p{2cm}|}
		\hline
		\multicolumn{7}{|c|}{SCED cost drivers} \\
		\hhline{|=======|}
		SCED descriptors & 75\% of nominal & 85\% of nominal & 100\% of nominal & 130\% of nominal & 160\% of nominal & \\
		\hline
		Rating level & Very low & Low & Nominal & High & Very high & Extra high \\
		\hline
		Effort multipliers & 1.43 & 1.14 & 1.00 & 1.00 & 1.00 & n/a \\
		\hline
	\end{tabular}
\end{table}

\subsubsection{Results}
Overall our results are expressed in the following table:

\begin{table}[H]
	\centering
	\begin{tabular}{|l|r|r|}
		\hline
		Cost Driver & Factor & Value \\
		\hline
		Required software reliability (RELY) & High & 1.10
\\
		Database size (DATA) & High & 1.14 \\
		Product complexity (CPLX) & Very high & 1.34
\\
		Required reusability (RUSE) & Nominal & 1.00
\\
		Documentation match to life-cycle needs (DOCU) & Nominal & 1.00
\\
		Execution time constraint (TIME) & Very high & 1.29
\\
		Main storage constraint (STOR) & High & 1.11
\\
		Platform volatility (PVOL) & Nominal & 1.00
\\
		Analyst capability (ACAP) & High & 0.85
\\
		Programmer capability (PCAP) & High & 0.88
\\
		Application experience (APEX) & High & 0.88 \\
		Platform Experience (PLEX) & High & 0.91
\\
		Language and Tool Experience (LTEX) & High & 0.91
\\
		Personnel continuity (PCON) & Very low & 1.12
\\
		Usage of Software Tools (TOOL) & High & 0.90
\\
		Multisite development (SITE) & Very high & 0.86
\\
		Required development schedule (SCED) & High & 1.00 \\
		\hline
		\multicolumn{2}{|l|}{Total} & 1.13694 \\
		\hline	
	\end{tabular}
\end{table}

\subsection{Effort equation}
This final equation gives us the effort estimation measured in Person-Months (PM):
\[ Effort = A \times EAF \times KSLOC^E \]
where:
\begin{nospaceflalign*}
	& A = 2.94 \textnormal{ (for COCOMO II)} \phantom{\sum} \\
	& EAF = 1.13694 \textnormal{ (product of all cost drivers)} \\
	& E = B + 0.01 \times \sum_{i} SF[i] = B + 0.01 \times 11.84 = 0.91 + 0.1184 = 1.0284 \\
	& \textnormal{(exponent derived from the scale drivers, with B = 0.91 for COCOMO II)} \\
\end{nospaceflalign*}

With this parameters we can compute the effort value, which has a lower
bound of:
\begin{nospaceflalign*}
	Effort = A \times EAF \times KSLOC^E & = 2.94 \times 1.13694 \times 6.783^{1.0284} \\
	& = 23.94\textnormal{ PM} \approx 24\textnormal{ PM}
\end{nospaceflalign*}
and an upper bound of:
\begin{nospaceflalign*}
	Effort = A \times EAF \times KSLOC^E & = 2.94 \times 1.13694 \times 9.709^{1.0284} \\
	& = 34.61\textnormal{ PM} \approx 35\textnormal{ PM}
\end{nospaceflalign*}

\subsection{Schedule estimation}
Regarding the final schedule, we are going to use the following formula:
\begin{nospaceflalign*}
	Duration = 3.67 \times Effort^F
\end{nospaceflalign*}
where:
\begin{nospaceflalign*}
	& F = 0.28 + 0.2 \times (E - B) = 0.28 + 0.2 \times 0.1184 = 0.31368 \\
\end{nospaceflalign*}
As a lower bound, we consider
\begin{nospaceflalign*}
	& Effort = 23.77\textnormal{ PM}
\\
	& Duration = 3.67 \times 23.77^{0.31368} = 9.91\textnormal{ months}
\\
\end{nospaceflalign*}
while as an upper bound, we consider
\begin{nospaceflalign*}
	& Effort = 34.61\textnormal{ PM}
\\
	& Duration = 3.67 \times 34.61^{0.31368} = 11.15\textnormal{ months}
\end{nospaceflalign*}