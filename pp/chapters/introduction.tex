\chapter{Introduction}

\section{Revision history}
\begin{table}[h]
	\centering
	\begin{tabular}{|c|c|c|c|}
		\hline
		\textbf{Version} & \textbf{Date} & \textbf{Authors} & \textbf{Summary} \\
		\hline
		1.0 & 22/01/2017 & Fabiani, Manivannan, Pozzolini & Initial release  \\
		\hline
	\end{tabular}

	\caption{Changelog of this document}
	\label{tab:revision_history}
\end{table}

\section{Purpose and scope}
The Project Plan (PP) document is intended to describe the best strategies for the management of PowerEnJoy with regards to all the aspects of the project, such as costs, schedule of the activities, resource allocation and effort estimation.

The product described is PowerEnJoy, a car-sharing service which offers to its users exclusively electric cars. It includes the common functionalities of its category: permitting to registered users to obtain the position of all the available cars, reserving one within a certain amount of time and continuously displaying the up-to-the-minute cost of the ride are just few of them. Moreover, PowerEnJoy stimulates users to behave virtuously towards the ecosystem by applying various types of discounts under specific conditions.

\section{Definitions, acronyms, abbreviations}
\begin{itemize}
	\item \emph{ACAP}: Analyst Capability
	\item \emph{APEX}: Applications Experience
	\item \emph{API}: Application Programming Interface
	\item \emph{BCE}: Business Controller Entity
	\item \emph{Car}: electric vehicle provided by the service
	\item \emph{CPLX}: Product Complexity
	\item \emph{DB}: Database
	\item \emph{DBMS}: Database Management System
	\item \emph{DD}: Design Document
	\item \emph{DOCU}: Documentation Match to Life-Cycle Needs
	\item \emph{ER}: Entity-Relationship
	\item \emph{GPS}: Global Positioning System
	\item \emph{Guest} or \emph{Guest user}: person not registered to the service
	\item \emph{ITPD}: Integration Test Plan Document
	\item \emph{LTEX}: Language and Tool Experience
	\item \emph{MVC}: Model View Controller
	\item \emph{OS}: Operating System, related both to desktop and mobile platforms
	\item \emph{PCAP}: Programmer Capability
	\item \emph{PCON}: Personnel Continuity
	\item \emph{PIN}: Personal Identification Number
	\item \emph{PLEX}: Platform Experience
	\item \emph{PP}: Project Plan
	\item \emph{PVOL}: Platform Volatility
	\item \emph{RASD}: Requirements Analysis and Specification Document
	\item \emph{Registered user}: see \emph{User}
	\item \emph{RELY}: Required Software Reliability
	\item \emph{REST}: Representational State Transfer
	\item \emph{RESTful}: that follows the REST principles
	\item \emph{RUSE}: Developed for Reusability
	\item \emph{Safe area}: set of parking spots where a user can leave a car without penalization 
	\item \emph{STOR}: Main Storage Constraint
	\item \emph{User}: person with a valid driving license registered to the service
	\item \emph{UX}: User eXperience
	\item \emph{W3C}: World Wide Web Consortium
\end{itemize}

\section{Reference documents}
The PP document has been composed following the guidelines reported in the Requirements Analysis and Specification Document delivered for this project. Moreover, the part describing the cost estimation follows the indications described in the second revision of the procedural software cost estimation model named Constructive Cost Model (COCOMO II), developed by Barry W.~Boehm.

With regards to the course named Software Engineering 2 and held by professors Luca Mottola and Elisabetta Di Nitto (Politecnico di Milano, a.~y. 2016/17), the document conforms to the guidelines provided during the lectures and within the material of the course.