\chapter{Introduction}

\section{Purpose}
The Requirement Analysis and Specifications Document aims to provide in detail every aspect of the service PowerEnJoy, including its components, goals, constraints, functional and non-functional requirements. Use cases and scenarios for all the users involved will be provided as well to conform the product's objectives to the real world.

The last part of this document is reserved to the formalization of some features of the system involving the utilization of Alloy, a declarative specification language which provides a structural modeling tool based on first-order logic.

The high-level functionalities described in the RASD are intended for both developers and project managers. The former have to implement and test the functionalities while the latter must examine whether every requirement has been respected. It may also be useful to users in order to best take advantage of the service.

\section{Description of the problem}
The product described in this RASD is PowerEnJoy, a car-sharing service which offers to its users exclusively electric cars. It includes the common functionalities of its category: permitting to registered users to obtain the position of all the available cars, reserving one within a certain amount of time and continuously displaying the up-to-the-minute cost of the ride are just few of them. Moreover, PowerEnJoy stimulates users to behave virtuously towards to the ecosystem by applying various type of discounts under specific conditions.

There are four software components that constitute the PowerEnJoy project. First, a back-end server provides APIs in order to simplify the communication related to the interactions of a user with the cars. Then two applications are available for a user to allow him/her the employment of every functionality: a web-based application (intended for visualization from desktop) and a mobile one. Lastly, every vehicle will be equipped with an on-board computer, used by the driver to manage the ride with the available options and see real-time information related to it, such as the time spent, the distance traveled and the total amount.

\section{Goals}
\begin{enumerate}[label={[G\arabic*]},labelindent=\parindent,leftmargin=*]
	\item \label{goal:registration} Registration of a user to the system
	\item \label{goal:cars_location} Finding the locations of the available cars
	\item \label{goal:reservation} Reservation of a car
	\item \label{goal:expiration} Expiration of reservation and penalization
	\item \label{goal:entry} Entry of registered user into the car
	\item \label{goal:charging} Start charging and notifying the registered user
	\item \label{goal:car_locking} Stop charging the registered user and lock the car
	\item \label{goal:safe_areas} Safe areas for parking the reserved cars
	\item \label{goal:passengers} Detection of extra passengers and applying discount
	\item \label{goal:battery} Detection of the battery status and applying discount
	\item \label{goal:special_areas} Detection of special parking areas and applying discount
	\item \label{goal:constraints} Checking parking and battery constraints and penalization
\end{enumerate}

\section{Domain properties}
\begin{itemize}
	\item User's data are always valid
	\item Location reported by the GPS is always accurate
	\item Every user can reserve just a car per time
\end{itemize}

\section{Glossary}

\subsection{Definitions}
\begin{itemize}
	\item \emph{Car}: electric vehicle provided by the service
	\item \emph{Guest} or \emph{Guest user}: person not registered to the service
	\item \emph{Registered user}: see \emph{User}
	\item \emph{Safe area}: set of parking spots where a user can leave a car without penalization 
	\item \emph{User}: person with a valid driving license registered to the service
\end{itemize}

\subsection{Acronyms}
\begin{itemize}
	\item \emph{API}: Application Programming Interface
	\item \emph{GPS}: Global Positioning System
	\item \emph{OS}: Operating System, related both to desktop and mobile platforms
	\item \emph{PIN}: Personal Identification Number
	\item \emph{RASD}: Requirements Analysis and Specification Document
	\item \emph{W3C}: World Wide Web Consortium
\end{itemize}

\section{Constrains}

\subsection{Regulatory policies}
While waiting for future conventions, at the moment toll and handicap parkings are forbidden. Timed parkings are also forbidden, since the user cannot ensure compliance with the deadline once left the car.

During the registration the system receive the user's permission to get his position and it has to handle sensible data according to the privacy law. To avoid SPAM the system can only use messages and notifications if strictly required to the proper operation of the system.

\subsection{Hardware limitations}
\begin{itemize}
	\item User's mobile device:
\begin{itemize}
	\item Connection speed \(\geqslant\) 3G
	\item GPS
	\item Enough memory available to install the app
\end{itemize}
	\item Car:
	\begin{itemize}
	\item GPS
	\item Weight sensor for each seat
	\item Fast Internet connection
	\item On-board computer with integrated system
\end{itemize}
\end{itemize}

\subsection{Interfaces to other applications}
+++++ TO DISCUSS TOGETHER +++++

\subsection{Parallel operation}
+++++ TO DISCUSS TOGETHER +++++

\section{Reference documents}
The Requirements Analysis and Specification Document has been composed following the indications and examples reported in the document ISO/IEC/ IEEE 29148, released by W3C, containing provisions for the processes and products related to the engineering of requirements for systems and software products and services throughout the life cycle.

With regards to the course named Software Engineering II and held by professors Luca Mottola and Elisabetta Di Nitto (Politecnico di Milano, a.~y. 2016/17), the document conforms to the guidelines provided during the lectures and within the material of the course.