\chapter{Introduction}

\section{Revision history}
\begin{table}[h]
	\centering
	\begin{tabular}{|c|c|c|c|}
		\hline
		\textbf{Version} & \textbf{Date} & \textbf{Authors} & \textbf{Summary} \\
		\hline
		1.0 & 15/01/2017 & Fabiani, Manivannan, Pozzolini & Initial release  \\
		\hline
	\end{tabular}

	\caption{Changelog of this document}
	\label{tab:revision_history}
\end{table}

\section{Purpose and scope}
The Integration Test Plan Document (ITPD) is intended to indicate the necessary tests in order to verify that all the components of the previously described system are properly integrated. This process is critically important to ensure that the unit-tested modules interact correctly.

The product described is PowerEnJoy, a car-sharing service which offers to its users exclusively electric cars. It includes the common functionalities of its category: permitting to registered users to obtain the position of all the available cars, reserving one within a certain amount of time and continuously displaying the up-to-the-minute cost of the ride are just few of them. Moreover, PowerEnJoy stimulates users to behave virtuously towards the ecosystem by applying various types of discounts under specific conditions.

\section{Definitions, acronyms, abbreviations}
\begin{itemize}
	\item \emph{API}: Application Programming Interface
	\item \emph{BCE}: Business Controller Entity
	\item \emph{Car}: electric vehicle provided by the service
	\item \emph{DB}: Database
	\item \emph{DBMS}: Database Management System
	\item \emph{DD}: Design Document
	\item \emph{ER}: Entity-Relationship
	\item \emph{GPS}: Global Positioning System
	\item \emph{Guest} or \emph{Guest user}: person not registered to the service
	\item \emph{ITPD}: Integration Test Plan Document
	\item \emph{MVC}: Model View Controller
	\item \emph{OS}: Operating System, related both to desktop and mobile platforms
	\item \emph{PIN}: Personal Identification Number
	\item \emph{RASD}: Requirements Analysis and Specification Document
	\item \emph{Registered user}: see \emph{User}
	\item \emph{REST}: REpresentational State Transfer
	\item \emph{RESTful}: that follows the REST principles
	\item \emph{Safe area}: set of parking spots where a user can leave a car without penalization 
	\item \emph{User}: person with a valid driving license registered to the service
	\item \emph{UX}: User eXperience
	\item \emph{W3C}: World Wide Web Consortium
\end{itemize}

\section{Reference documents}
The Integration Test Plan Document has been composed following the indications and examples reported in the document 

It also based on the previous document delivered for this project: the Requirements Analysis and Specification Document, describing fundamental aspects of PowerEnJoy such as domain assumptions, goals, functional and non-functional requirements, and the Design Document

With regards to the course named Software Engineering 2 and held by professors Luca Mottola and Elisabetta Di Nitto (Politecnico di Milano, a.~y. 2016/17), the document conforms to the guidelines provided during the lectures and within the material of the course.